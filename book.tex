\documentclass[11pt,a4paper]{book}
\usepackage{amssymb}
\usepackage{amsthm}
\usepackage{fontspec, xunicode, xltxtra}
\usepackage{titlesec}
\usepackage{indentfirst}
\usepackage{enumerate}
\usepackage{multicol}
\usepackage{multirow}
\usepackage[fleqn]{amsmath}
\usepackage{xcolor}
\usepackage{listings}
\usepackage{graphicx}
\usepackage{fancyhdr}
%\usepackage{algorithm}
%\usepackage{algorithmic}
%\usepackage{amsmath}
%\usepackage{sectsty}
%\usepackage{picinpar}
%\usepackage{bm}
%\usepackage[boxed,linesnumbered]{algorithm2e}


%\input{config.tex}

\defaultfontfeatures{Mapping=tex-text} % to support TeX conventions like ``---''

%\setsansfont{Deja Vu Sans}
%\setmonofont{Deja Vu Mono}

% other LaTeX packages.....
%\usepackage{geometry} % See geometry.pdf to learn the layout options. There are lots.
%\geometry{a4paper} % or letterpaper (US) or a5paper or....
%\usepackage[parfill]{parskip} % Activate to begin paragraphs with an empty line rather than an indent

\usepackage{graphicx} % support the \includegraphics command and options

%Fonts
%\setCJKmainfont[BoldFont={Adobe Heiti Std}, ItalicFont={Adobe Kaiti Std}]{Adobe Song Std}
%\setCJKmonofont{Adobe Fangsong Std}
\setmainfont[Mapping=tex-text, BoldFont={Minion Pro Bold}]{Minion Pro}
%\setsansfont{Liberation Sans}
%\setmonofont{Bitstream Vera Sans Mono}
%\punctstyle{kaiming}
%\setCJKfamilyfont{kai}{KaiTi}
%\setCJKfamilyfont{hei}{SimHei}
%\setCJKmainfont{SimSun}



%PageStyle
\setlength{\parindent}{2.5em}
\pagestyle{fancy}
\fancyhead{}
\fancyfoot{} % clear all fields
%\fancyfoot[LF]{Copyright by Yinyanghu}
\fancyfoot[RF]{\thepage}
\renewcommand{\footrulewidth}{0.4pt}
\renewcommand{\headrulewidth}{0.0pt}




%NewCommands
\newcommand\abs[1]{\left\lvert #1 \right\rvert}
\newcommand\floor[1]{\left\lfloor #1 \right\rfloor}
\newcommand\ceil[1]{\left\lceil #1 \right\rceil}
\newcommand\yin[1]{\textit{#1}}
\newcommand\yang[1]{\textbf{#1}}
%\newcommand{\kai}{\CJKfamily{kai}}
%\newcommand{\hei}{\CJKfamily{hei}}
%\newcommand\TT{\rule{0pt}{2.6ex}}
%\newcommand\BB{\rule[-1.2ex]{0pt}{0pt}}



%Enumerate
\renewcommand{\labelenumi}{\bfseries{\alph{enumi}.}}
\renewcommand{\labelenumii}{\bfseries{\arabic{enumii}.}}


\title{Penetration Testing and Ethical Hacking}
\author{Many authors}
%\institude{Nanjing University, China}
\date{\today}


\begin{document}

%[frontmatter] turns off chapter numbering and uses roman numerals for page numbers;
%[mainmatter] turns on chapter numbering, resets page numbering and uses arabic numerals for page numbers;
%[appendix] resets chapter numbering, uses letters for chapter numbers and doesn't fiddle with page numbering;
%[backmatter] turns off chapter numbering and doesn't fiddle with page numbering.

\frontmatter
\maketitle

\tableofcontents
\listoffigures
\listoftables

\mainmatter

\chapter{Preface}

This book is based on lecture notes for \textit{TX6244:Ethical Hacking and Penetration Testing}, a course offered by Center for Cybersecurity(UKM) since 2014. This course is jointly developed by Cybersecurity Malaysia and Center for Cybersecurity. People who are involves in teaching and developing the lectures materials are (in no particular order);

\begin{quote}
Mohd Zamri Murah,
\end{quote}


We hope this book will provide a good overview for students into the field of penetration testing and ethical hacking. We also hope these materials will encourage students to further explore this exciting field.

\chapter{Introduction}





\nocite{*}

\section{Tools}

\subsection{Linux}
\subsection{Kali Linux}
\subsection{git}
\subsection{Python}
\subsection{Raspberry Pi}
\section{Cybersecurity News}
\section{Definitions}
\section{Web resources}
\subsection{github.com}
\subsection{Web Sites}

\chapter{Preliminary}
\section{Network}
\section{How Internet works}
\section{IP, DNS}
\section{Web servers}



\chapter{Network}

\section{Reconnaissance}
\section{Scanning}
\section{Enumeration}
\section{Vulnerability Assessment}



\chapter{Web Apps}
\section{Reconnaissance}
\section{Scanning}
\section{Enumeration}
\section{Vulnerability Assessment}

\chapter{Wireless}



\chapter{Mobile}



\bibliography{buku}
\bibliographystyle{unsrt}

\end{document}
